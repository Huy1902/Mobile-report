\clearpage
\setcounter{chapter}{2}
\chapter[{ĐẶC TẢ YÊU CẦU}]{ĐẶC TẢ YÊU CẦU}

\label{chapter:requirements}

Phân tích đặc tả yêu cầu là bước quan trọng trong quá trình phát triển phần mềm, nhằm xác định và mô tả các yêu cầu chức năng và phi chức năng của hệ thống. Đây là 
nền tảng để thiết kế, triển khai và kiểm thử phần mềm. Trong chương này, chúng tôi sẽ tập trung trình bày các yêu cầu chức năng và phi chức năng của hệ thống ở 
\ref{sec:user_requirements} và minh
họa chúng thông qua các sơ đồ và bảng biểu cụ thể ở \ref{sec:use_case}, \ref{sec:use_case_specification} và \ref{sec:activity}.


\clearpage 
\section{Thu thập và phân tích đặc tả yêu cầu}
\label{sec:user_requirements}
\subsection{Yêu cầu người sử dụng}


\begin{itemize}
	\item Có thể đăng nhập vào hệ thống an toàn.
	\item Có thể phát được bài hát.
	\item Có thể thêm bài hát yêu thích.
	\item Có thể đăng ký tài khoản mới.
	\item Có thể lấy lại mật khẩu khi quên.
	\item Có thể xem bài hát phát gần đây
	\item Có thể trò chuyện cùng chatbot
	\item Có thể nhận bài hát gợi ý
	\item Có thể sửa thông tin cá nhân
\end{itemize}
\subsection{Yêu cầu hệ thống}
\subsubsection{Yêu cầu chức năng}
\begin{itemize}
	\item Người dùng có thể đăng nhập bằng email và mật khẩu, thông qua Google
	\item Người dùng đăng ký được tài khoản mới
	\item Khi quên mật khẩu, người dùng có thể lấy lại
	\item Người dùng phát được bài hát
	\item Người dùng mở lại xem được những bài hát gần đây
	\item Người dùng trò chuyện được với chatbot và được hỗ trợ
	\item Người dùng sửa thông tin cá nhân khi có nhu cầu
	\item Người dùng nhận những bài hát được gợi ý
	\item Khi thích, người dùng lưu lại được bài hát
\end{itemize}
\subsubsection{Yêu cầu phi chức năng}
\begin{itemize}
	\item Hiệu suất: Hệ thống xử lý yêu cầu hiệu quả, phản hồi nhanh, chịu tải tốt
	\item Bảo mật: Có cơ chế bảo mật riêng, bảo mật thông tin người dùng
	\item Độ chính xác: hệ thống gợi ý và chatbot có khả năng phản hồi chính xác
\end{itemize}
\input{chapter/chapter3/sec3.2.tex}
\input{chapter/chapter3/sec3.3.tex}
\section{Biểu đồ hoạt động}
Phần Activity Diagram (Biểu đồ hoặt động) được sử dụng nhằm mo tả chi tiết về luồng hoạt động trong
hệ thống, từ khi người dùng bắt đầu một hoạt động cho đến cách hệ thống phản hồi kết quả cuối cùng. 

Biểu đò này giúp nhóm phát triển hiểu rõ hơn các bước nghiệp vụ và tương tác giữa người dùng và hệ thống
, đặc biệt trong các ca sử dụng (Use Case) quan trọng. Thông qua biểu đồ này, ta có thể quan sát được toàn bộ 
tiến trình xử lý, bao gồm các hành động mà người dùng thực hiện, các xử lý nội bộ của hệ thống, điều kiện rẽ nhánh, 
cũng như các kết thúc có thể xảy ra. Điều này không chỉ hỗ trợ việc phân tích yêu cầu mà còn giúp lập trình viên 
nắm bắt logic nghiệp vụ để triển khai chính xác hơn.

\textit{Chú thích: Decision trong Activity Diagram theo chuẩn thì được biểu diễn bằng một hình thoi, tuy nhiên do code bằng PlantUml, mặc định decision 
là hình lục giác nên Decision trong tài liệu này được để là một hình lục giác. }

\subsection{Đăng ký} 
Activity Diagram "Đăng ký" mô tả quy trình mà người dùng thực hiện để tạo tài khoản 
mới trên hệ thống. Biểu đồ này thể hiện các hoạt động chính như nhập thông tin đăng ký, kiểm tra
tính hợp lệ của dữ liệu, xử lý các trường hợp lỗi và kết thúc bằng việc tạo tài khoản thành công. 

Biểu đồ giúp minh họa rõ các hoạt động giữa người dùng và hệ thống, cho phép nhìn tổng thể các bước,
các quyết định quan trọng cũng như tương tác giữa các đối tượng. 
\begin{figure}[H]
	\centering
	\includegraphics[width=1\textwidth]{figures/register-activity.png}
	\caption{Biểu đồ hoạt động chức năng "Đăng ký"}
\end{figure}

\subsection{Đăng nhập}
Activity Diagram "Đăng nhập" mô tả quy trình mà người dùng thực hiện để đăng nhập 
vào hệ thống. Biểu đồ này thể hiện các hoạt động chính như nhập thông tin đăng nhập, kiểm tra
tính hợp lệ của dữ liệu, xử lý các trường hợp lỗi và kết thúc bằng việc tạo đăng nhập thành công. 

Biểu đồ giúp minh họa rõ các hoạt động giữa người dùng và hệ thống, cho phép nhìn tổng thể các bước,
các quyết định quan trọng cũng như tương tác giữa các đối tượng. 
\begin{figure}[H]
	\centering
	\includegraphics[width=1\textwidth]{figures/login-activity.png}
	\caption{Biểu đồ hoạt động chức năng "Đăng nhập"}
\end{figure}

\subsection{Đăng xuất}
Activity Diagram "Đăng xuất" mô tả quy trình mà người dùng thực hiện để đăng xuất 
khỏi hệ thống. Biểu đồ này thể hiện các hoạt động chính như tương tác của người dùng với hệ thống
tính hợp lệ của dữ liệu, xử lý các trường hợp lỗi và kết thúc bằng việc đăng xuất thành công. 

Biểu đồ giúp minh họa rõ các hoạt động giữa người dùng và hệ thống, cho phép nhìn tổng thể các bước,
các quyết định quan trọng cũng như tương tác giữa các đối tượng. 
\begin{figure}[H]
	\centering
	\includegraphics[width=1\textwidth]{figures/logout-activity.png}
	\caption{Biểu đồ hoạt động chức năng "Đăng xuất"}
\end{figure}

\subsection{Tìm kiếm bài hát}
Activity Diagram "Tìm kiếm bài hát" mô tả quy trình mà người dùng thực hiện để tìm kiếm bài hát muốn nghe 
trên hệ thống. Biểu đồ này thể hiện các hoạt động chính như nhập cách người dùng thao tác với giao diện, nhập thông tin cần tìm kiếm, 
hệ thống xử lý dữ liệu, xử lý các trường hợp lỗi và kết thúc bằng việc tạo tài khoản tìm thấy một loạt danh sách bài hát thành công. 

Biểu đồ giúp minh họa rõ các hoạt động giữa người dùng và hệ thống, cho phép nhìn tổng thể các bước,
các quyết định quan trọng cũng như tương tác giữa các đối tượng. 
\begin{figure}[H]
	\centering
	\includegraphics[width=1\textwidth]{figures/searchsongs-activity.png}
	\caption{Biểu đồ hoạt động chức năng "Tìm kiếm bài hát"}
\end{figure}




\phantomsection
\chapter*{Kết chương}
Chương \ref{chapter:requirements} cho chúng ta cái nhìn tổng quan về các yêu cầu chức năng và phi chức năng của hệ thống \textbf{InsightTune}. Chúng tôi đã trình bày
các yêu cầu này thông qua biểu đồ ca sử dụng, biểu đồ hoạt động và bảng đặc tả yêu cầu. Những yêu cầu này sẽ là cơ sở cho thiết kế hệ thống của chương 
\ref{chapter:architecture} tiếp theo.