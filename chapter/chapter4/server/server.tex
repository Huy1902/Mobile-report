\subsection{Server}
\label{subsection:server}

Phần trước chúng tôi đã diễn giải và phân tích Client, ở phần này chúng tôi xin nói về thành phần quan trọng còn lại trong 
kiến trúc Client - Server, Server.

Trong kiến trúc Client - Server, Server có vai trò cốt lõi như sau:
\begin{itemize}
  \item Là cửa ngõ tiếp nhận và phản hồi yêu cầu
  \item Thực thi nghiệp vụ (Business Logic)
  \item Truy cập cơ sở dữ liệu và bảo toàn nhất quán dữ liệu
  \item Quản lý phiên và bảo mật xác thực danh tính
  \item Kiểm soát lưu lượng
  \item Ghi lại (logging) hoạt động, trạng thái của hệ thống
\end{itemize}

So với Client yêu cầu dung lượng nhẹ, khả năng dễ triển khai tương thích với nhiều loại sản phẩm. Server cần có khả năng mở rộng
cao khi hệ thống sẽ ngày càng phức tạp trong tương lai. Với các mô hình monolithic truyền thống, việc triển khai và xây dựng sẽ
nhanh chóng dễ dàng giai đoạn đầu. Tuy nhiên, càng về sau, khi hệ thống trở nên đồ sộ và bắt đầu cần mở rộng, monolithic lại 
cho thấy những hạn chế của mình trong khả năng mở rộng và tái sử dụng code.

Do đó, để đáp ứng yêu cầu \ref{req:scale} đưa ra về khả năng mở rộng, chúng tôi đã lựa chọn kiến trúc microservice. Một kiến trúc
chia hệ thống thành nhiều dịch vụ nhỏ, độc lập triển khai. Kiến trúc chung của microservice có thể minh họa qua hình \ref{fig:microservice}
sau đây.
\begin{figure}[H]
	\centering
	\includegraphics[width=1\textwidth]{figures/microservice.png}
	\caption{Kiến trúc microservice chung}
	\label{fig:microservice}
\end{figure}

Ở trong kiến trúc microservice điểm đặc biệt nằm ở các dịch vụ nhỏ của nó. Mỗi dịch vụ (service) phải thỏa mãn các yêu cầu cố 
định, tạo nên sự linh hoạt và khả năng mở rộng cao đặc trưng của kiến trúc này. Các yêu cầu này có thể tóm gọn lại như sau:
\begin{itemize}
  \item Có miền nghiệp vụ rõ ràng (bounded context).
  \item Tự quản lý dữ liệu riêng (database-per-service).
  \item Giao tiếp qua API (REST, gRPC,...) hoặc broker (kafka, rabbitMQ).
  \item Có vòng đời phát triển và triển khai tách biệt với các dịch vụ khác.
\end{itemize}



\begin{figure}[H]
  \centering
  \includegraphics[width=1\textwidth]{figures/server.png}
  \caption{Kiến trúc server bậc cao}
  \label{fig:server_architecture}
\end{figure}

\begin{figure}[H]
  \centering
  \includegraphics[width=1\textwidth]{figures/spring.png}
  \caption{Kiến trúc 3 lớp của Spring khi xây dựng Microservice}
  \label{fig:spring_architecture}
\end{figure}