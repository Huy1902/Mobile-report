\section{Người phản biện: Trần Quang Đỉnh}
Về tổng quan(Summary): Đóng góp chính của dự án trong giải quyết vấn đề thực tế:
\begin{itemize}
  \item Tự động hóa quy trình báo cáo thanh toán hàng tháng (scan, lập hóa đơn tự động)
  \item Tạo sản phẩm kết nối chủ nhà người thuê (dành cho chủ nhà và dành cho khách) AI của dự án: Triển khai YOLO để quét điện nước, 
  Chatbot và knowledge graph để trả lời câu hỏi về luật
\end{itemize}

\vspace{2cm}
Về \textbf{điểm mạnh (Strengths)}:
\begin{itemize}
  \item Có tính thực tiễn, giải quyết thực trạng nhà ở tại các thành phố lớn tại Việt Nam hiện nay. 
  \item Áp dụng công nghệ AI, triển khai tích hợp thành công công nghệ của LLM và CV vào hệ thống, bám sát nhu cầu thực tiễn, giúp cải thiện trải nghiệm người dùng. 
  \item Đầy đủ tính năng, quy trình nghiệp vụ chặt chẽ. 
  \item Quy trình kiểm thử phù hợp với chức năng. 
  \item Cấu trúc báo cáo khoa học, rõ ràng. 
\end{itemize}

\vspace{2cm}
Về \textbf{điểm yếu (Weaknesses)}:
\begin{itemize}
    \item Phần thu thập yêu cầu lan man, chưa tập trung làm rõ yêu cầu của hệ thống
    \item Sai kí hiệu các biểu đồ.
    \item Đặc tả yêu cầu chưa nhất quán về tên gọi, hình vẽ. 
    \item Các ca kiểm thử chưa có ngày và người kiểm thử. 
\end{itemize}

\vspace{2cm}
Tôi xin được \textbf{chấm điểm} đề tài như sau:
\begin{itemize}
  \item Về \textbf{chất lượng (quality)}: độ hoàn thiên của báo cáo cao, tuy nhiên mô tả nghiệp vụ bài toán bằng biểu đồ hoạt động còn sơ sài, đôi chỗ lam man thừa nhiều chi tiết.
\textbf{8/10}
  \item Về \textbf{tính rõ ràng (clarity)}: Còn những lỗi chính tả trong phiên bản review; 
\textbf{9/10}
  \item Về \textbf{độ ảnh hưởng (significance)}: Bài toán thực tế, động lực rõ ràng. 
Giải pháp chưa có tính ứng dụng, ổn về mặt ý tưởng. 
\textbf{8.5/10}
  \item Về \textbf{độ độc đáo (originality)}: 
\textbf{9/10}
\end{itemize}

Tôi xin được đánh giá số điểm \textbf{8.5/10}