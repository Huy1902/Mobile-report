\section{Người phản biện: Trần Quang Đỉnh}
Về tổng quan(Summary): Đóng góp chính của dự án trong giải quyết vấn đề thực tế:
\begin{itemize}
  \item Tự động hóa quy trình báo cáo thanh toán hàng tháng (scan, lập hóa đơn tự động)
  \item Tạo sản phẩm kết nối chủ nhà người thuê (dành cho chủ nhà và dành cho khách) AI của dự án: Triển khai YOLO để quét điện nước, 
  Chatbot và knowledge graph để trả lời câu hỏi về luật
\end{itemize}

\vspace{2cm}
Về \textbf{điểm mạnh (Strengths)}:
\begin{itemize}
  \item Có tính thực tiễn, giải quyết thực trạng nhà ở tại các thành phố lớn tại Việt Nam hiện nay. 
  \item Áp dụng công nghệ AI, triển khai tích hợp thành công công nghệ của LLM và CV vào hệ thống, bám sát nhu cầu thực tiễn, giúp cải thiện trải nghiệm người dùng. 
  \item Đầy đủ tính năng, quy trình nghiệp vụ chặt chẽ. 
  \item Quy trình kiểm thử phù hợp với chức năng. 
  \item Cấu trúc báo cáo khoa học, rõ ràng. 
\end{itemize}

\vspace{2cm}
Về \textbf{điểm yếu (Weaknesses)}:
\begin{itemize}
    \item Phần thu thập yêu cầu lan man, chưa tập trung làm rõ yêu cầu của hệ thống
    \item Sai kí hiệu các biểu đồ.
\item Các ca kiểm thử chưa có ngày và người kiểm thử. 
\end{itemize}

Kết luận về điểm yếu: Thiết kế các biểu đồ còn nhiều bất cập, các ca kiểm thử còn thiếu thông tin.  

\vspace{2cm}
Tôi xin được \textbf{chấm điểm} đề tài như sau:
\begin{itemize}
  \item Về \textbf{chất lượng (quality)}: Độ hoàn thiện của báo cáo là cao. Việc thiết kế các ca sử dụng chỉn chu, các ca kiểm thử được thiết kế tương xứng. 
  Giao diện thiết kế tốt, tuân thủ nguyên tắc
  YOLO và KG đều được xây dựng ổn, tuy nhiên chưa có độ chính xác cao.
\textbf{9/10}
  \item Về \textbf{tính rõ ràng (clarity)}: Còn những lỗi chính tả trong phiên bản review; biểu đồ hoạt động không rõ ràng, 
  phần kiểm thử có những case fail nhưng kỳ vọng thực tế giống nhau. Cấu trúc các phần rõ ràng
\textbf{8/10}
  \item Về \textbf{độ ảnh hưởng (significance)}: Bài toán thực tế, động lực rõ ràng. 
Giải pháp chưa có tính ứng dụng, ổn về mặt ý tưởng. 
Chatbot khó ứng dụng được, thiết kế hệ thống và đánh giá đều cần cải tiến.
\textbf{8/10}
  \item Về \textbf{độ độc đáo (originality)}: Auto billing và scan không mới nhưng tích hợp vào một ứng dụng nhà trọ tại Việt Nam là độc đáo.
\textbf{9/10}
\end{itemize}
Đi kèm với những \textbf{câu hỏi} sau:
\begin{itemize}
  \item Có thể thiết kế Knowledge Graph có thêm phụ thuộc giữa các tác nhân trong luật được không?
  \item Phần thiết kế hệ thống đơn giản nhất có thể làm một server side theo monolithic architecture để wrap lại hết cả 3 dịch vụ trong thiết kế. Tại sao lại để client gọi trực tiếp cả 3 như vậy?
  \item YOLO ra F1 0.57 cho ứng dụng có vẻ không cao? Đây là vấn đề dữ liệu hay giới hạn của mô hình?
  \item Faithful cho một ứng dụng liên quan tới luật là khá thấp. Kể cả thấp, nếu thêm một ablation study về việc không sử dụng Graph với có sẽ khá ấn tượng. Điều này có khả thi?
\end{itemize}

Tôi xin được đánh giá số điểm \textbf{8.5/10} với đánh giá cuối cùng như sau:
Nói chung, hệ thống khá ấn tượng. Về thiết kế hệ thống và khâu kiểm thử cần cải thiện thêm. UI của nhóm rất tốt. Dự án trên Github được làm chỉn chu.
Ý tưởng hay, phân tích đặc tả yêu cầu tốt, tích hợp AI có kiểm thử, tính ứng dụng chưa cao.