\chapter{Giới thiệu}

Nhìn chung, phần giới thiệu được nhóm Táo trình bày khá hoàn thiện, có đầy đủ các mục cần thiết. Hai phần kết quả thực nghiệm và cấu trúc báo cáo tốt. Một vài vấn 
đề tồn tại trong phần đặt vấn đề cần được nhóm lưu ý để cải thiện.

\section{Đặt vấn đề, giới thiệu chung}
Về \textbf{điểm mạnh}, nhóm đã có những nghiên cứu cụ thể về đề tài, tìm ra những vấn đề thực tế còn tồn đọng để giải quyết. Từ đó, rút ra được động lực rõ
ràng cho đề tài.

Tuy nhiên về \textbf{điểm yếu}:
\begin{itemize}
  \item Nhóm đặt ra nhiều vấn đề liên tục khiến người đọc khó nắm bắt. Từ đoạn 1 tới đoạn 3, nhóm liên tục đưa ra các vấn đề khác nhau theo 
cấu trúc song song không luận điểm hay dẫn dắt. Sau đó những vấn đề này không được tóm gọn lại, mà nhóm quyết định rút ra hai vấn đề từ khảo sát một số ứng dụng 
trên thị trường. 
  \item Trình bày vấn đề không rõ ràng. Ví dụ, trong hình \ref{fig:problem2}, nhóm trình bày việc "triển khai góc nhìn ... từng tháng", điều này khá rời rạc khỏi
vấn đề 2.
  \item Hai vấn đề nhóm in đậm lại chưa liên kết tới phần chatbot mà nhóm xây dựng. Đây là một điểm khá quan trọng ảnh hưởng đóng góp của nhóm, nhưng nhóm lại 
nhắc tới mà không nêu bật.
\end{itemize}

\begin{figure}[H]
  \centering
  \includegraphics[width=0.7\textwidth]{figures/problem2.png}
  \caption{Đoạn vấn đề 2 trong phần đặt vấn đề}
  \label{fig:problem2}
\end{figure}

\section{Kết quả thực nghiệm}

Phần này là điểm mạnh của dự án khi nhóm đã có khách hàng dùng thử.

\section{Cấu trúc báo cáo}

Báo cáo được nhóm trình bày rõ ràng, có cấu trúc hệ thống. Tính cấu trúc của báo cáo được nhóm thể hiện rõ qua việc nhóm có mục lục chi tiết, 
các chương mục được đánh số rõ ràng, định dạng thống nhất. Nội dung của chúng được trình bày rõ ràng trong phần này.

